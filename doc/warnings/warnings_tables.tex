\section{Warning Tables}

\subsection{Function Warnings}



Local warnings are warnings that can be detected through a local analysis of
one function.



\noindent\\\begin{tabular}{|c|p{8cm}|c|} \hline
Identifier & Title & Implemented      \\ \hline
  1 & variable not initialized & 0.1 \\ \hline
  2 & unused function argument & 0.1 \\ \hline
  3 & duplicate function argument & 0.2 \\ \hline
  4 & duplicate return variable   & 0.2 \\ \hline
  5 & function argument used as return variable   & 0.2 \\ \hline
  6 & return variable is never set & 0.2 \\ \hline
  7 & return variable used as a local variable & 0.2 \\ \hline
  0 & variable defined but not used &  \\ \hline
  0 & use of ``exec'' function      &  \\ \hline
  0 & for variable modified in for loop &  \\ \hline
  0 & for bounds modified during for loop &  \\ \hline
  0 & keyword used as a variable &  \\ \hline
  0 & use of return/resume should be avoided &  \\ \hline
  0 & default value for argument never used &  \\ \hline
  0 & variable is not initialized in all paths &  \\ \hline
  0 & field of tlist not defined &  \\ \hline
\end{tabular}

\subsection{File Warnings}

File warnings are warnings that can be detected through a local
analysis of one file. In such an analysis, it is supposed that the
identifiers of functions defined in the file can only be used to call
these functions.

\noindent\\\begin{tabular}{|c|p{8cm}|c|} \hline
Identifier & Title & Implemented      \\ \hline
  0 & duplicate function identifier &  \\ \hline
  0 & too many arguments in function call &  \\ \hline
  0 & too few arguments in function call  &  \\ \hline
  0 & redefinition of primitive function &  \\ \hline
  0 & redefinition of standard library function &  \\ \hline
  0 & labeled argument does not exist &  \\ \hline
  0 & toplevel expression in file &  \\ \hline
\end{tabular}

\subsection{Global Warnings}

Global warnings are warnings that can be detected through a global
analysis of a whole project, i.e. knowning all the files used in the
project.

\noindent\\\begin{tabular}{|c|p{8cm}|c|} \hline
Identifier & Title & Implemented      \\ \hline
  0 & undefined variable or function &  \\ \hline
\end{tabular}

\subsection{Typing Warnings}



\noindent\\\begin{tabular}{|c|p{8cm}|c|} \hline
Identifier & Title & Implemented      \\ \hline
  0 & variable with different types &  \\ \hline
  0 & matrix is growing in loop &  \\ \hline
  0 & matrix with different dimensions &  \\ \hline
\end{tabular}

\subsection{Style Warnings}



Style warnings are warnings caused by not following the style conventions.



See \url{http://wiki.scilab.org/Code%20Conventions%20for%20the%20Scilab%20Programming%20Language}



\noindent\\\begin{tabular}{|c|p{8cm}|c|} \hline
Identifier & Title & Implemented      \\ \hline
601 & functions should start with lowercase &  \\ \hline
602 & functions should not contain numeric values &  \\ \hline
\end{tabular}

\section{Function Warnings}



Local warnings are warnings that can be detected through a local analysis of
one function.



\subsection{W001 --- variable not initialized}


\begin{verbatim}
function z = f()
  z = cos(x)  // W001: "x" not initialized
endfunction
\end{verbatim}


\subsection{W002 --- unused function argument}


\begin{verbatim}
function z = f(y)// W002: "y" is not used
  z = cos(0)
endfunction
\end{verbatim}


\subsection{W003 --- duplicate function argument}




Two arguments in a function definition have the same name.



Note: it can only be an error !



Example:\begin{verbatim}
function z = f(a,b,a)// W003: argument "a" appears twice
  z = cos(a) + cos(b)
endfunction
\end{verbatim}




\subsection{W004 --- duplicate return variable  }




Two return variables in a function definition have the same name.



Note: it can only be an error !



Example:\begin{verbatim}
function [a,b,a] = f()  // W004: return variable "a" appears twice
  a = cos(0)
  b = cos(pi)
  c = cos(pi*2)
endfunction
\end{verbatim}




\subsection{W005 --- function argument used as return variable  }




A function argument appears as a return variable.



Example:\begin{verbatim}
function [a] = f(a)  // W005: return variable "a" is also an argument
  a = cos(a)
endfunction
\end{verbatim}




\subsection{W006 --- return variable is never set}




A return variable is never set in the function, whatever the path taken.



Example:\begin{verbatim}
function [a,b] = f()  // W006: return variable "a" is never set
  if( a > 0 ) then
    b = 1;
  else
    b = 0;
  end;
endfunction
\end{verbatim}
Note that, on this example, warnings W001 (variable ``a'' not initialized)
and W007 (return variable used as a variable) should also be displayed.



\subsection{W007 --- return variable used as a local variable}




A return variable is used in the function as a local variable: return
variables should only be assigned values, never read.



Example:\begin{verbatim}
function [a] = f()
  a = 0;
  for i = 1:100,
    a = a +1;   // W007: return variable "a" used as a local variable
  end;
endfunction
\end{verbatim}




\subsection{W000 --- variable defined but not used}




\subsection{W000 --- use of ``exec'' function     }




\subsection{W000 --- for variable modified in for loop}




\subsection{W000 --- for bounds modified during for loop}




\subsection{W000 --- keyword used as a variable}




\subsection{W000 --- use of return/resume should be avoided}




\subsection{W000 --- default value for argument never used}




\subsection{W000 --- variable is not initialized in all paths}




\subsection{W000 --- field of tlist not defined}




\section{File Warnings}

File warnings are warnings that can be detected through a local
analysis of one file. In such an analysis, it is supposed that the
identifiers of functions defined in the file can only be used to call
these functions.

\subsection{W000 --- duplicate function identifier}




The same name is used by two definitions of functions in two different
files.



\subsection{W000 --- too many arguments in function call}




A call to a function defined in the same file provides too many
arguments, compared to the number specified in the function definition.



Extra arguments will usually not be used. Changing the behavior of a
function depending on its number of arguments is usually a bad
practice.



Example:\begin{verbatim}
function [c] = max(a,b)
  if( a > b ) then
    c = a;
  else
    c = a;
  end;
endfunction;
x = max(1,2,3);  // W 201 : "max" used with too many arguments
\end{verbatim}




\subsection{W000 --- too few arguments in function call }




A call to a function defined in the same file does not provide all the
arguments specified in the function definition.



Changing the behavior of a function depending on its number of
arguments is usually a bad practice.



Example:\begin{verbatim}
function [c] = max(a,b)
  if( a > b ) then
    c = a;
  else
    c = a;
  end;
endfunction;
x = max(1);  // W 202 : "max" used with too few arguments
\end{verbatim}




\subsection{W000 --- redefinition of primitive function}




\subsection{W000 --- redefinition of standard library function}




\subsection{W000 --- labeled argument does not exist}




\subsection{W000 --- toplevel expression in file}




\section{Global Warnings}

Global warnings are warnings that can be detected through a global
analysis of a whole project, i.e. knowning all the files used in the
project.

\subsection{W000 --- undefined variable or function}




\section{Typing Warnings}



\subsection{W000 --- variable with different types}




\subsection{W000 --- matrix is growing in loop}




\subsection{W000 --- matrix with different dimensions}




\section{Style Warnings}



Style warnings are warnings caused by not following the style conventions.



See \url{http://wiki.scilab.org/Code%20Conventions%20for%20the%20Scilab%20Programming%20Language}



\subsection{W601 --- functions should start with lowercase}




\subsection{W602 --- functions should not contain numeric values}




