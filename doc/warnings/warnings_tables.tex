\section{Warning Tables}

\subsection{Function Warnings}



Local warnings are warnings that can be detected through a local analysis of
one function.



\noindent\\\begin{tabular}{|c|p{8cm}|c|} \hline
Identifier & Title & Implemented      \\ \hline
W001 & variable not initialized & 0.1 \\ \hline
W002 & unused function argument & 0.1 \\ \hline
W003 & duplicate function argument & 0.2 \\ \hline
W004 & duplicate return variable   & 0.2 \\ \hline
W005 & function argument used as return variable   & 0.2 \\ \hline
W006 & return variable is never set & 0.2 \\ \hline
W007 & return variable used as a local variable & 0.2 \\ \hline
W008 & for variable modified in for loop & 0.3 \\ \hline
W009 & redefinition of primitive function & 0.3 \\ \hline
W010 & redefinition of local function & 0.3 \\ \hline
W011 & redefinition of library function & 0.3 \\ \hline
W012 & unexpected string argument & 0.3 \\ \hline
W013 & primitive with too many arguments & 0.3 \\ \hline
W000 & variable defined but not used &  \\ \hline
W000 & variable redefined before being used &  \\ \hline
W000 & use of ``execstr'' function      &  \\ \hline
W000 & for bounds modified during for loop &  \\ \hline
W000 & keyword used as a variable &  \\ \hline
W000 & use of return/resume to define variables should be avoided &  \\ \hline
W000 & variable is not initialized in all paths &  \\ \hline
W000 & field of tlist not defined &  \\ \hline
W000 & break outside of loop &  \\ \hline
W000 & continue outside of loop &  \\ \hline
W000 & labeled argument provided twice &  \\ \hline
W000 & unlabeled argument after labeled argument &  \\ \hline
W000 & missing parenthesis for function call &  \\ \hline
W000 & non void return value of function not used &  \\ \hline
\end{tabular}

\subsection{File Warnings}

File warnings are warnings that can be detected through a local
analysis of one file. In such an analysis, it is supposed that the
identifiers of functions defined in the file can only be used to call
these functions.

\noindent\\\begin{tabular}{|c|p{8cm}|c|} \hline
Identifier & Title & Implemented      \\ \hline
W000 & duplicate function identifier &  \\ \hline
W000 & type not defined in redefinition &  \\ \hline
W000 & too many arguments in function call &  \\ \hline
W000 & too few arguments in function call  &  \\ \hline
W000 & labeled argument does not exist &  \\ \hline
W000 & toplevel expression in file &  \\ \hline
W000 & labeled argument in varargin &  \\ \hline
\end{tabular}

\subsection{Global Warnings}

Global warnings are warnings that can be detected through a global
analysis of a whole project, i.e. knowning all the files used in the
project.

\noindent\\\begin{tabular}{|c|p{8cm}|c|} \hline
Identifier & Title & Implemented      \\ \hline
W000 & undefined variable or function &  \\ \hline
\end{tabular}

\subsection{Typing Warnings}



\noindent\\\begin{tabular}{|c|p{8cm}|c|} \hline
Identifier & Title & Implemented      \\ \hline
W000 & variable with different types &  \\ \hline
W000 & matrix is growing in loop &  \\ \hline
W000 & matrix with different dimensions &  \\ \hline
\end{tabular}

\subsection{Style Warnings}



Style warnings are warnings caused by not following the style conventions.



See \url{http://wiki.scilab.org/Code%20Conventions%20for%20the%20Scilab%20Programming%20Language}



\noindent\\\begin{tabular}{|c|p{8cm}|c|} \hline
Identifier & Title & Implemented      \\ \hline
W601 & functions should start with lowercase &  \\ \hline
W602 & functions should not contain numeric values &  \\ \hline
\end{tabular}

\section{Function Warnings}



Local warnings are warnings that can be detected through a local analysis of
one function.



\subsection{W001 --- variable not initialized}


\begin{verbatim}
function z = f()
  z = cos(x)  // W001: "x" not initialized
endfunction
\end{verbatim}


Remarks : referencing global variables should be avoided, variables should be passed as arguments. This variable can also have been introduced by a 'resume' in a function called from the current context, or an 'execstr' function



\subsection{W002 --- unused function argument}


\begin{verbatim}
function z = f(y)// W002: "y" is not used
  z = cos(0)
endfunction
\end{verbatim}


\subsection{W003 --- duplicate function argument}




Two arguments in a function definition have the same name.



Note: it can only be an error !



Example:\begin{verbatim}
function z = f(a,b,a)// W003: argument "a" appears twice
  z = cos(a) + cos(b)
endfunction
\end{verbatim}




\subsection{W004 --- duplicate return variable  }




Two return variables in a function definition have the same name.



Note: it can only be an error !



Example:\begin{verbatim}
function [a,b,a] = f()  // W004: return variable "a" appears twice
  a = cos(0)
  b = cos(pi)
  c = cos(pi*2)
endfunction
\end{verbatim}




\subsection{W005 --- function argument used as return variable  }




A function argument appears as a return variable.



Example:\begin{verbatim}
function [a] = f(a)  // W005: return variable "a" is also an argument
  a = cos(a)
endfunction
\end{verbatim}




However, this is often used as an optimization to avoid
a useless copy when returning a matrix.



\subsection{W006 --- return variable is never set}




A return variable is never set in the function,
whatever the path taken.



Example:\begin{verbatim}
function [a,b] = f()  // W006: return variable "a" is never set
  if( a > 0 ) then
    b = 1;
  else
    b = 0;
  end;
endfunction
\end{verbatim}
Note that, on this example, warnings W001
(variable ``a'' not initialized)
and W007 (return variable used as a variable) should also be displayed.



\subsection{W007 --- return variable used as a local variable}




A return variable is used in the function as a local variable:
 return variables should only be assigned values, never read.



Example:\begin{verbatim}
function [a] = f()
  a = 0;
  for i = 1:100,
    a = a +1;   // W007: return variable "a" used as a local variable
  end;
endfunction
\end{verbatim}




However, this is often used as an optimization to avoid
a useless copy when returning a matrix.



\subsection{W008 --- for variable modified in for loop}




Modifying the variable of a 'for' loop does not change the
loop behavior.

\begin{verbatim}
for i=1:100
   i=i+2; // W: modifying variable of 'for' loop does not change loop behavior
   disp(i);
end
\end{verbatim}


\subsection{W009 --- redefinition of primitive function}




Redefining a primitive is usually a bad idea

\begin{verbatim}
function disp(x) // W009: redefinition of primitive 'disp'
  disp(x+1)
endfunction
\end{verbatim}


\subsection{W010 --- redefinition of local function}




The code is redefining a function that was already defined in this scope.

\begin{verbatim}
function f()
  function pr(x), disp(x),endfunction
  function pr(x,y) // W010: redefinition of local function 'pr'
    disp(x,y)
  endfunction
endfunction

\end{verbatim}


\subsection{W011 --- redefinition of library function}




The code is redefining a function that was already defined in the toplevel scope.

\begin{verbatim}
function pr(x), disp(x),endfunction
function f()
   function pr(x,y) // W010: redefinition of library function 'pr'
     disp(x,y)
   endfunction
endfunction

\end{verbatim}


\subsection{W012 --- unexpected string argument}




Some functions expect a limited list of strings as flags. If another string is passed, it is usually an error.

\begin{verbatim}
function f()
   execstr("x=1", "x=2"); // W012: "x=2" instead of "catcherr"
endfunction
\end{verbatim}


\subsection{W013 --- primitive with too many arguments}




Some primitives expect a limited number of arguments. Providing more arguments will usually trigger an error at runtime.

\begin{verbatim}
function f(a,b)
   disp(strcat('a=',a,'b=',b); // W013: 'strcat' expects fewer arguments
endfunction
\end{verbatim}


\subsection{W000 --- variable defined but not used}


\begin{verbatim}
function f()
  x = 1;  // W : variable "x" defined but not used
  y = 2;
  disp(y);
endfunction
\end{verbatim}


The variable might be used in another function, but it is
a bad practice to pass values like that instead of by argument.



\subsection{W000 --- variable redefined before being used}


\begin{verbatim}
function f()
  x = 1;  // W : variable "x" redefined before being used
  x = 2;
  disp(x);
endfunction
\end{verbatim}


\subsection{W000 --- use of ``execstr'' function     }


\begin{verbatim}
x = execstr(instr, 'catcherr'); // W: avoid use of 'execstr' 
\end{verbatim}


\subsection{W000 --- for bounds modified during for loop}




Modifying the bounds of a loop does not change the behavior of
the 'for' loop

\begin{verbatim}
a = 3;
for i=1:a
  disp(i);
  a = 5; // W: modifying bound of 'for' loop does not change loop behavior
end
\end{verbatim}


\subsection{W000 --- keyword used as a variable}




Keywords should not be used as names of variables.

\begin{verbatim}
function f(if) // W : 'if' should not be used as a variable
  disp(if)
endfunction
\end{verbatim}


\subsection{W000 --- use of return/resume to define variables should be avoided}


\begin{verbatim}
function x_is_one()
  x = resume(1) // W : 'resume' should not be used
endfunction
\end{verbatim}


\subsection{W000 --- variable is not initialized in all paths}


\begin{verbatim}
function f(a,b)
  if (a > 3) then
    x = b
  end
  disp(x) // W: 'x' is not initialized in all paths
endfunction
\end{verbatim}


\subsection{W000 --- field of tlist not defined}


\begin{verbatim}
function f()
  x = tlist(["obj", "a","b"])
  x.c = 3 // W : "c" was not declared inside the tlist declaration
endfunction
\end{verbatim}


\subsection{W000 --- break outside of loop}


\begin{verbatim}
function f()
  x = 1
  break; // W : break outside of loop
  disp(x);
endfunction
\end{verbatim}


\subsection{W000 --- continue outside of loop}


\begin{verbatim}
function f()
  x = 1
  continue; // W : continue outside of loop
  disp(x);
endfunction
\end{verbatim}


\subsection{W000 --- labeled argument provided twice}


\begin{verbatim}
function f(a, b, c) endfunction
f(a, b, b=3) // W : argument "b" provided twice
\end{verbatim}


\subsection{W000 --- unlabeled argument after labeled argument}


\begin{verbatim}
f(a, b=2, c) // W : unlabeled argument after labeled argument "b"
\end{verbatim}


\subsection{W000 --- missing parenthesis for function call}


\begin{verbatim}
[a,b] = f // W : you should use "f()" instead of "f"
\end{verbatim}


\subsection{W000 --- non void return value of function not used}


\begin{verbatim}
function [x]=f() x=1; endfunction
f(); // W : non void return value of function not used

\end{verbatim}


\section{File Warnings}

File warnings are warnings that can be detected through a local
analysis of one file. In such an analysis, it is supposed that the
identifiers of functions defined in the file can only be used to call
these functions.

\subsection{W000 --- duplicate function identifier}




The same name is used by two definitions of functions in
two different files.



\subsection{W000 --- type not defined in redefinition}


\begin{verbatim}

  x = tlist(["object", "a","b"])
function %objcet_p(mytlist) // 'objcet was not defined as a type
  mprintf("a=%s\n", x.a);
endfunction
\end{verbatim}


\subsection{W000 --- too many arguments in function call}




A call to a function defined in the same file provides too many
arguments, compared to the number specified in the function definition.



Extra arguments will usually not be used. Changing the behavior
of a function depending on its number of arguments is usually a bad
practice.



Example:\begin{verbatim}
function [c] = max(a,b)
  if( a > b ) then
    c = a;
  else
    c = a;
  end;
endfunction;
x = max(1,2,3);  // W 201 : "max" used with too many arguments
\end{verbatim}




\subsection{W000 --- too few arguments in function call }




A call to a function defined in the same file does not provide all the
arguments specified in the function definition.



Changing the behavior of a function depending on its number of
arguments is usually a bad practice.



Example:\begin{verbatim}
function [c] = max(a,b)
  if( a > b ) then
    c = a;
  else
    c = a;
  end;
endfunction;
x = max(1);  // W 202 : "max" used with too few arguments
\end{verbatim}




\subsection{W000 --- labeled argument does not exist}


\begin{verbatim}
function f(a, b)
endfunction
f(1, c=3)
\end{verbatim}


\subsection{W000 --- toplevel expression in file}




It is better to avoid execution of expressions in files, and
to call an initialization function at the beginning.



\subsection{W000 --- labeled argument in varargin}




Labeled arguments are not allowed in varargin

\begin{verbatim}
function f(a, b, varargin)
endfunction
f(1,2, c=3)
\end{verbatim}


\section{Global Warnings}

Global warnings are warnings that can be detected through a global
analysis of a whole project, i.e. knowning all the files used in the
project.

\subsection{W000 --- undefined variable or function}




\section{Typing Warnings}



\subsection{W000 --- variable with different types}




\subsection{W000 --- matrix is growing in loop}




\subsection{W000 --- matrix with different dimensions}




\section{Style Warnings}



Style warnings are warnings caused by not following the style conventions.



See \url{http://wiki.scilab.org/Code%20Conventions%20for%20the%20Scilab%20Programming%20Language}



\subsection{W601 --- functions should start with lowercase}




\subsection{W602 --- functions should not contain numeric values}




