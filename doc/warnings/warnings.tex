\documentclass{article}

\usepackage{url}
\usepackage{todonotes} %package texlive-latex-extra


\title{Scilint Warnings}
\author{INRIA \& OCamlPro}

\begin{document}

\maketitle

\section{Warning Tables}

% update these tables everytime a new warning is added

\subsection{Function Warnings}

Local warnings are warnings that can be detected through a local analysis of
one function.

\begin{tabular}{|l|l|l|} \hline
Identifier & Title & Roadmap          \\ \hline
W001 & variable not initialized & 0.1 \\ \hline
W002 & unused function argument & 0.1 \\ \hline
W003 & duplicate function argument &  \\ \hline
W004 & duplicate return variable   &  \\ \hline
W005 & function argument used as return variable   &  \\ \hline
W006 & return variable is never set &  \\ \hline
W007 & return variable used as a local variable &  \\ \hline
% W--- & variable defined but not used & \\ \hline
% W--- & use of ``exec'' function      & \\ \jline
\end{tabular}

\subsection{File Warnings}

File warnings are warnings that can be detected through a local
analysis of one file. In such an analysis, it is supposed that the
identifiers of functions defined in the file can only be used to call
these functions.

\begin{tabular}{|l|l|l|} \hline
Identifier & Title & Roadmap          \\ \hline
FW 001 & too many arguments in function call & \\ \hline
FW 002 & too few arguments in function call  & \\ \hline
% FW 003 & duplicate function identifier & \\ \hline
% FW 004 & redefinition of primitive function & \\ \hline
% FW 005 & redefinition of standard library function & \\ \hline
\end{tabular}

\subsection{Global Warnings}

Global warnings are warnings that can be detected through a global
analysis of a whole project, i.e. knowning all the files used in the
project.

\begin{tabular}{|l|l|l|} \hline
Identifier & Title & Roadmap          \\ \hline
GW 001  & duplicate function identifier & \\ \hline
\end{tabular}

\subsection{Style Warnings}

Style warnings are warnings caused by not following the style
conventions.

See \url{http://wiki.scilab.org/Code%20Conventions%20for%20the%20Scilab%20Programming%20Language}

\begin{tabular}{|l|l|l|} \hline
Identifier & Title & Roadmap          \\ \hline
SW 001  & functions should start with lowercase & \\ \hline
SW 002  & functions should not contain numeric values & \\ \hline
\end{tabular}

\section{Local warnings}

\subsection{W001 --- Variable Not Initialized}

\begin{verbatim}
function z = f()
  z = cos(x)  // W001: "x" not initialized
endfunction
\end{verbatim}

\subsection{W002 --- Unused Function Argument}

\begin{verbatim}
function z = f(y)// W002: "y" is not used
  z = cos(0)  
endfunction
\end{verbatim}

\subsection{W003 --- Duplicate Function Argument}

Two arguments in a function definition have the same name. 

Note: it can only be an error !

Example:
\begin{verbatim}
function z = f(a,b,a)// W003: argument "a" appears twice
  z = cos(a) + cos(b)
endfunction
\end{verbatim}

\subsection{W004 --- Duplicate Return Variable}

Two return variables in a function definition have the same name.

Note: it can only be an error !

Example:
\begin{verbatim}
function [a,b,a] = f()  // W004: return variable "a" appears twice
  a = cos(0)
  b = cos(pi)
  c = cos(pi*2)
endfunction
\end{verbatim}

\subsection{W005 --- Function argument used as return variable}

A function argument appears as a return variable.

Example:
\begin{verbatim}
function [a] = f(a)  // W005: return variable "a" is also an argument
  a = cos(a)
endfunction
\end{verbatim}

\subsection{W006 --- Return variable is never set}

A return variable is never set in the function, whatever the path taken.

Example:
\begin{verbatim}
function [a,b] = f()  // W006: return variable "a" is never set
  if( a > 0 ) then
    b = 1;
  else
    b = 0;
  end;
endfunction
\end{verbatim}
Note that, on this example, warnings W001 (variable ``a'' not initialized)
and W007 (return variable used as a variable) should also be displayed.


\subsection{W007 --- Return variable used as a variable}

A return variable is used in the function as a local variable: return
variables should only be assigned values, never read. 

Example:
\begin{verbatim}
function [a] = f()
  a = 0;
  for i = 1:100,
    a = a +1;   // W007: return variable "a" used as a local variable
  end;
endfunction
\end{verbatim}


\subsection{W--- --- X}


\section{File warnings}

\subsection{FW 001 --- Too Many Arguments in a Function Call}

A call to a function defined in the same file provides too many
arguments, compared to the number specified in the function definition.

Extra arguments will usually not be used. Changing the behavior of a
function depending on its number of arguments is usually a bad
practice.


Example:
\begin{verbatim}
function [c] = max(a,b)
  if( a > b ) then
    c = a;
  else
    c = a;
  end;
endfunction;
x = max(1,2,3);  // FW 001 : "max" used with too many arguments
\end{verbatim}

\subsection{FW 002 --- Too Few Arguments in a Function Call}

A call to a function defined in the same file does not provide all the
arguments specified in the function definition.

Changing the behavior of a function depending on its number of
arguments is usually a bad practice.

Example:
\begin{verbatim}
function [c] = max(a,b)
  if( a > b ) then
    c = a;
  else
    c = a;
  end;
endfunction;
x = max(1);  // FW 002 : "max" used with too few arguments
\end{verbatim}

\section{Global warnings}

\subsection{GW 001 --- Duplicate function identifier}

The same name is used by two definitions of functions in two different
files.

\end{document}
